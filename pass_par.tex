\begin{frame}
  \frametitle{Passagem de parâmetros}
  \begin{block}{}
    \begin{itemize}
    \item Em lua parâmetros são passados por valor.
    \item As referências dos parâmetros internos são mantidas.
    \item Valores do tipo table, function, thread e userdata são objetos
    \item Atribuição, passagem de parâmetro, e retorno de
          funções sempre lidam com referências para tais valores
    \item estas operações não implicam em qualquer espécie de cópia
    \end{itemize}
  \end{block}
\end{frame}

\begin{frame}[fragile]
  \frametitle{Passagem de parâmetros}
  \begin{block}{}
   \begin{lstlisting}
     a = { value = 0 }
   \end{lstlisting}
   \begin{block}{}
     O valor de a é um objeto (tabela). Se parâmetro de
     uma função *valor* será passado é uma
     referência.
   \end{block}
  \end{block}
\end{frame}

\begin{frame}[fragile]
  \frametitle{Passagem de parâmetros}
  \begin{lstlisting}
  function inc(param)
    param.value = param.value + 1
  end
  
  function inc(param)
    param = nil
  end
  \end{lstlisting}
\end{frame}

\begin{frame}[fragile]
  \frametitle{Passagem de parâmetros}
  \begin{block}{}
    Implementação da chamada em C:
    
    \begin{lstlisting}
    void lua_call (lua_State *L, int nargs, int nresults);
    \end{lstlisting}
    
    a função a ser chamada é empilhada na pilha;
    os argumentos da função são empilhados em ordem direta;
    lua_call é chamada 
    
    Todos os argumentos e o valor da função são desempilhados
    da pilha quando a função é chamada. 
    Os resultados da função são empilhados na pilha quando
    a função retorna. 
  \end{block}
\end{frame}
