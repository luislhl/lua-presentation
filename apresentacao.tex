\documentclass{ime-beamer}
\usepackage[portuges]{babel}
\usepackage[utf8]{inputenc}
\usepackage{graphicx}
\usepackage{listings}			% Para usar \lstinputlisting e incluir código
\usepackage{color}
\usepackage{multicol}

\definecolor{dkgreen}{rgb}{0,0.6,0}
\definecolor{gray}{rgb}{0.5,0.5,0.5}
\definecolor{mauve}{rgb}{0.58,0,0.82}

\lstset{frame=tb,
  language=[5.2]Lua,
  showstringspaces=false,
  columns=flexible,
  basicstyle={\tiny\ttfamily},
  numbers=none,
  numberstyle=\tiny\color{gray},
  keywordstyle=\color{blue},
  commentstyle=\color{dkgreen},
  stringstyle=\color{mauve},
  breaklines=true,
  breakatwhitespace=true,
  tabsize=3
}

\title[Lua]{%
  Lua
}
\author[Luis Helder\and Victor Bramigk]{%
  Luis Helder\\
  Victor Bramigk\\
}

% as imagens ficam nesse diretório
\graphicspath{{img/}}

\begin{document}

\frame{\maketitle}

\frame{%
  \frametitle{Roteiro}
  \tableofcontents
}

\section{Introdução}
\frame{%
  \frametitle{O que é}
  \begin{block}{}
    \centering
    \begin{itemize}
      \item Lua é uma linguagem de extensão projetada para suportar programação
        procedural.
      \item Concebida em 1993 na PUC-Rio, é a única LP desenvolvida num país
        em desenvolvimento a alcançar relevância global.
      \item Como uma linguagem de extensão, Lua necessita de um programa
        anfitrião. Porém, é oferecido um interpretador que funciona
        como programa anfitrião.
      \item Também oferece suporte à programação funcional e à orientada a
        objetos.
      \item É implementada como uma biblioteca escrita em C.
    \end{itemize}
  \end{block}
}

\frame{%
  \frametitle{Método de Implementação}
  \begin{block}{}
      \centering
      \begin{itemize}
        \item Lua é uma linguagem híbrida.
        \item Compilação por padrão é em tempo de execução, mas pode ser feita
          previamente, para aumentar a performance.
        \item VM baseada em registro.
      \end{itemize}
  \end{block}
}

\section{Amarrações}
\begin{frame}[fragile]
  \frametitle{Escopo}
  \begin{block}{}
      \centering
      \begin{itemize}
        \item Toda variável é global, a não ser que seja explicitamente
          declarada como local. 
        \item Utiliza-se 'local' para declaração de variáveis locais.
      \end{itemize}
  \end{block}
  \begin{block}{}
    \begin{lstlisting}
     x = 10                -- global variable
     do                    -- new block
       local x = x         -- new 'x', with value 10
       print(x)            --> 10
       x = x+1
       do                  -- another block
         local x = x+1     -- another 'x'
         print(x)          --> 12
       end
       print(x)            --> 11
     end
     print(x)              --> 10  (the global one)
    \end{lstlisting}
  \end{block}
\end{frame}

\section{Valores e Tipos de Dados}
\frame{%
  \frametitle{Tipos}
  \begin{block}{}
    Lua é dinamicamente tipada. 
    Não há definições de tipos na linguagem, os valores carregam seu próprio tipo.
  
    A linguagem define 8 tipos de dados:
    \begin{itemize}
      \item nil
      \item boolean
      \item number (integer/float)
      \item string
      \item function
      \item userdata (bloco de dados do programa em C)
      \item thread
      \item table (único tipo composto)
    \end{itemize}
  \end{block}
}

\frame{%
  \frametitle{Tables}
  \begin{block}{}
    Implementam a ideia de array associativo, ou seja, associam chaves e
        valores. As chaves podem ser qualquer valor menos nil e NaN.

    São o único mecanismo de estrutura de dados em Lua.

    Podem ser usadas para criar:
    \begin{itemize}
      \item Arrays
      \item Records
      \item Sets
      \item Trees
      \item Namespaces
      \item Classes
      \item Etc.
    \end{itemize}
  \end{block}
}

\begin{frame}[fragile]
  \frametitle{Tables - exemplos}
  \begin{block}{}
    \begin{itemize}
      \item Table usado como um Record: 
        \begin{lstlisting}
    point = { x = 10, y = 20 }   -- Create new table
    print(point["x"])            -- Prints 10
    print(point.x)               -- Has exactly the same meaning as line above. The easier-to-read dot notation is just syntactic sugar.
        \end{lstlisting}
      \item Table usado como namespace:
        \begin{lstlisting}
    Point = {}
     
    Point.new = function(x, y)
      return {x = x, y = y}  --  return {["x"] = x, ["y"] = y}
      end
       
      Point.set_x = function(point, x)
        point.x = x  --  point["x"] = x;
        end
        \end{lstlisting}
    \end{itemize}
  \end{block}
\end{frame}

\frame{%
  \frametitle{Metatables}
  \begin{block}{}
    \begin{itemize}
      \item Controlam como objetos se comportam ao serem submetidos a determinados
    eventos, como operações aritméticas, comparações e indexações, ou ao ser
    \emph{garbage collected}.
      \item Uma metatable é uma table associada ao objeto, onde as chaves são
        o nome do evento, e os valores são funções, chamadas de
        \emph{metamethods}, a
        serem chamadas na ocorrência do evento.
      \item No caso de operações binárias, Lua procura no primeiro operando um \emph{metamethod}
        definido para o evento correspondente. Caso não encontre, procura no
        segundo operando.
    \end{itemize}
  \end{block}
}

\frame{%
  \frametitle{Metatables - exemplos de eventos}
  \begin{block}{}
    \begin{itemize}
      \item add - Ocorre ao aplicar-se o operador + ao objeto.
      \item eq - Ocorre ao aplicar-se o perador == ao objeto. 
      \item index - Se o objeto for uma \emph{Table}, ocorre ao tentar-se
        acessar um índice inexistente. Caso contrário, ao tentar-se acessar
        acessar qualquer índice. O \emph{metamethod} pode ser tanto uma
        função quanto uma Table.
      \item call - Ocorre caso o objeto não seja uma \emph{function} e
        tenta-se aplicar o operador de chamada () nele.
    \end{itemize}
  \end{block}
}

\begin{frame}[fragile]
  \frametitle{Metatables - exemplo de uso}
  \begin{block}{}
    Fibonacci em Lua usando-se Tables e Metatables
    \begin{lstlisting}
    fibs = { 1, 1 }                                -- Initial values for fibs[1] and fibs[2].
    setmetatable(fibs, {
      __index = function(values, n)                --[[ __index is a function predefined by Lua, 
                                                        it is called if key "n" does not exist. ]]
        values[n] = values[n - 1] + values[n - 2]  -- Calculate and memorize fibs[n].
        return values[n]
      end
    })
    print("Posição 6: ", fibs[6])
    print("Posição 200: ", fibs[200])
    \end{lstlisting}
  \end{block}
\end{frame}

\section{Variáveis e Constantes}

\section{Expressões e Comandos}
\begin{frame}[fragile]
  \frametitle{Comandos - Atribuição}
  \begin{block}{}
    \begin{itemize}
      \item Lua suporta atribuição múltipla:
        \begin{lstlisting}
          x, y = 2, 3
        \end{lstlisting}
      \item Todas as expressões no comando são avaliadas antes das atribuições serem
        realizadas. O código a seguir altera a[3], não a[4]:
        \begin{lstlisting}
          i = 3
          i, a[i] = i+1, 20
        \end{lstlisting}
      \item Funções podem ser chamadas na lista de valores à direita:
        \begin{lstlisting}
        function f()
          return 2,3
        a,b,c = 1, f()
        \end{lstlisting}
    \end{itemize}
  \end{block}{}
\end{frame}

\begin{frame}[fragile]
  \frametitle{Comandos - Estruturas de controle}
  \begin{block}{}
    \begin{lstlisting}
    if condition then
      --statements
    elseif condition then
      --statements
    else
      --statements
    end

    while condition do
      --statements
    end
     
    repeat
      --statements
    until condition
       
    for i = first,last,delta do     --delta may be negative, allowing the for loop to count down
      -- break or goto exit the loop
    end

    for key, value in pairs(_G) do
      print(key, value)
      -- break exit the loop
    end
    \end{lstlisting}
  \end{block}
\end{frame}

\section{Exemplos de Algoritmos}
\begin{frame}[fragile]
  \frametitle{Bubblesort}
  \begin{lstlisting}
  function bubbleSort(A)
    local itemCount=#A
    local hasChanged
    repeat
      hasChanged = false
      itemCount=itemCount - 1
      for i = 1, itemCount do
        if A[i] > A[i + 1] then
          A[i], A[i + 1] = A[i + 1], A[i]
          hasChanged = true
        end
      end
    until hasChanged == false
  end

  list = { 5, 6, 1, 2, 9, 14, 2, 15, 6, 7, 8, 97 }
  bubbleSort(list)
  for i, j in pairs(list) do
    print(j)
  end
  \end{lstlisting}
\end{frame}

\frame{%
  \frametitle{Checagem de tipos}
  \begin{block}
    \centering
    \begin{itemize}
      \item Tipagem Dinâmica
      \item Valores nunca são tratados com tipo incorreto
      \item Verificação de tipos em tempo de execução
    \end{itemize}
  \end{block}
}

\begin{frame}
  \frametitle{Passagem de parâmetros}
  \begin{block}{}
    \begin{itemize}
    \item Em lua parâmetros são passados por valor.
    \item As referências dos parâmetros internos são mantidas.
    \item Valores do tipo table, function, thread e userdata são objetos
    \item Atribuição, passagem de parâmetro, e retorno de
          funções sempre lidam com referências para tais valores
    \item estas operações não implicam em qualquer espécie de cópia
    \end{itemize}
  \end{block}
\end{frame}

\begin{frame}[fragile]
  \frametitle{Passagem de parâmetros}
  \begin{block}{}
   \begin{lstlisting}
     a = { value = 0 }
   \end{lstlisting}
   \begin{block}{}
     O valor de a é um objeto (tabela). Se parâmetro de
     uma função *valor* será passado é uma
     referência.
   \end{block}
  \end{block}
\end{frame}

\begin{frame}[fragile]
  \frametitle{Passagem de parâmetros}
  \begin{lstlisting}
  function inc(param)
    param.value = param.value + 1
  end
  
  function inc(param)
    param = nil
  end
  \end{lstlisting}
\end{frame}

\begin{frame}[fragile]
  \frametitle{Passagem de parâmetros}
  \begin{block}{}
    Implementação da chamada em C:
    
    \begin{lstlisting}
    void lua_call (lua_State *L, int nargs, int nresults);
    \end{lstlisting}
    
    a função a ser chamada é empilhada na pilha;
    os argumentos da função são empilhados em ordem direta;
    lua\_call é chamada 
    
    Todos os argumentos e o valor da função são desempilhados
    da pilha quando a função é chamada. 
    Os resultados da função são empilhados na pilha quando
    a função retorna. 
  \end{block}
\end{frame}

\frame{%
  \frametitle{Tipos abstratos de dados}
  \begin{block}{}
  \begin{lstlisting}
local Vector = {}
Vector.__index = Vector
 
function Vector:new(x, y, z)    -- The constructor
  return setmetatable({x = x, y = y, z = z}, Vector)
end
 
function Vector:magnitude()     -- Another method
  -- Reference the implicit object using self
  return math.sqrt(self.x^2 + self.y^2 + self.z^2)
end
 
local vec = Vector:new(0, 1, 0) -- Create a vector
print(vec:magnitude())          -- Call a method (output: 1)
print(vec.x)                    -- Access a member variable (output: 0)
  \end{lstlisting}
  \end{block}
}

\begin{frame}[fragile]
  \frametitle{Polimorfismo}
  \begin{block}{}
    \centering
    Formas: coerção, sobrecarga e paramétrico.\\

    A coerção é feita quando se opera um operando de um tipo e um valor
    de outro tipo é fornecido, desde que esse tipo possa ser convertido.
    Exemplo:
    
  \begin{block}{}
    \begin{lstlisting}
        print (2 ^ "3") -- output: 6
    \end{lstlisting}
  \end{block}
    
    A string "3" é convertida para o inteiro 3,
    tornando a operação possível.
  \end{block}
\end{frame}

\begin{frame}[fragile]
  \frametitle{Polimorfismo}
  \begin{block}{}
    \centering
Não é necessário declarar o tipo dos parâmetros formais. 
Exemplo, a função table.insert adiciona um valor de qualquer
tipo a uma tabela (exceto nil).
  \end{block}
\end{frame}

\begin{frame}[fragile]
  \frametitle{Polimorfismo}
  \begin{block}{}
    \centering
    \begin{lstlisting}
    t = {}
    table.insert( t, "10" )
    table.insert( t, 10 )
    table.insert( t, function() return 10 end )
    for k, v in pairs( t ) do
       print( type(v), v )
    end
    -- Imprime:
    string 10
    number 10
    function     function: 0xa037158
    \end{lstlisting}
  \end{block}
\end{frame}

\frame{%
  \frametitle{Checagem de tipos}
  \begin{block}
Lua é uma linguagem dinamicamente 
tipada : o interpretador Lua sabe o tipo de 
cada valor, e verifica cada operação antes
de ser executada, para checar se os 
tipos dos operandos estão corretos
  \end{block}
}

\frame{%
  \frametitle{Sobrecarga}
  \begin{block}{}
    \centering
Todo valor em Lua pode ter uma metatabela. 
 metatabela: tabela que define o comportamento
do valor original com relação a certas operações especiais.
  \end{block}
}

\begin{frame}[fragile]
  \frametitle{Sobrecarga}
  \begin{block}{}
    \centering
Exemplo:
  \begin{lstlisting}
  function foo (o1, o2) 
    print( '__eq call' )
    return false 
  end
  
  t1 = {}
  setmetatable( t1, {__eq = foo} )
  
  t2 = t1
  print( t1 == t2 ) --> true
          -- string '__eq call' not printed
  
  t3 = {}
  setmetatable( t3, {__eq = foo} )
  if t1 == t3 then end  --> __eq call
          -- foo(...) was called
  \end{lstlisting}
  \end{block}
\end{frame}

\begin{frame}[fragile]
  \frametitle{Herança}
  \begin{block}{}
  \begin{lstlisting}
    Rectangle = { width = 0, height = 0 }

    Cuboid = Rectangle:new ({depth=0})
  \end{lstlisting}

   Para definir a função volume do Cuboid, fazemos:

  \begin{lstlisting}
    function Cuboid:volume ()

    return self.width * self.height * self.depth

    end
  \end{lstlisting}
  \end{block}
\end{frame}

\begin{frame}[fragile]
  \frametitle{Herança múltiplas}

  \begin{lstlisting}
    Button = {
    super = { Shape, EventResponser }
    }
  \end{lstlisting}
\end{frame}

\input{class_abstr}
%\frame{%
  \frametitle{Exceções}
  \begin{block}
    \centering
 Ações de Lua começam a partir de código C no programa hospedeiro
 que chama uma função da biblioteca de Lua (ver \textbf{lua\_pcall})
  \end{block}
  \begin{block}
    \centering
Sempre que um erro ocorre durante a compilação ou execução,
O controle retorna para C, que pode tomar as medidas apropriadas
  \end{block}
  \begin{block}
    \centering
O código Lua pode explicitamente gerar um erro através de uma chamada
à função error. É possível capturar erros com a função \textbf{pcall} 
  \end{block}
}



\end{document}
% vim: tw=80 et ts=2 sw=2 sts=2
