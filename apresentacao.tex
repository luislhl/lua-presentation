\documentclass{ime-beamer}
\usepackage[portuges]{babel}
\usepackage[utf8]{inputenc}
\usepackage{graphicx}
\usepackage{listings}			% Para usar \lstinputlisting e incluir código
\usepackage{color}
\usepackage{multicol}

\definecolor{dkgreen}{rgb}{0,0.6,0}
\definecolor{gray}{rgb}{0.5,0.5,0.5}
\definecolor{mauve}{rgb}{0.58,0,0.82}

\lstset{frame=tb,
  language=[5.2]Lua,
  aboveskip=3mm,
  belowskip=3mm,
  showstringspaces=false,
  columns=flexible,
  basicstyle={\tiny\ttfamily},
  numbers=none,
  numberstyle=\tiny\color{gray},
  keywordstyle=\color{blue},
  commentstyle=\color{dkgreen},
  stringstyle=\color{mauve},
  breaklines=true,
  breakatwhitespace=true,
  tabsize=3
}

\title[Computação em nuvem]{%
  Cloud Computiong
}
\author[Luis Helder\and Victor Bramigk]{%
  Luis Helder\\
  Victor Bramigk\\
}

% as imagens ficam nesse diretório
\graphicspath{{img/}}

\begin{document}

\frame{\maketitle}

\frame{%
  \frametitle{Roteiro}
  \tableofcontents
}

\section{Introdução}
\frame{%
  \frametitle{O que é}
  \begin{block}{}
    \centering
    \begin{itemize}
      \item Lua é uma linguagem de extensão projetada para suportar programação
        procedural.
      \item Concebida em 1993 na PUC-Rio, é a única LP desenvolvida num país
        em desenvolvimento a alcançar relevância global.
      \item Como uma linguagem de extensão, Lua necessita de um programa
        anfitrião. Porém, é oferecido um interpretador que funciona
        como programa anfitrião.
      \item Também oferece suporte à programação funcional e à orientada a
        objetos.
      \item É implementada como uma biblioteca escrita em C.
    \end{itemize}
  \end{block}
}

\frame{%
  \frametitle{Método de Implementação}
  \begin{block}{}
      \centering
      \begin{itemize}
        \item Lua é uma linguagem híbrida.
        \item Compilação por padrão é em tempo de execução, mas pode ser feita
          previamente, para aumentar a performance.
        \item VM baseada em registro.
      \end{itemize}
  \end{block}
}

\section{Amarrações}
\begin{frame}[fragile]
  \frametitle{Escopo}
  \begin{block}{}
      \centering
      \begin{itemize}
        \item Toda variável é global, a não ser que seja explicitamente
          declarada como local. 
        \item Utiliza-se 'local' para declaração de variáveis locais.
      \end{itemize}
  \end{block}
  \begin{block}{}
    \begin{lstlisting}
     x = 10                -- global variable
     do                    -- new block
       local x = x         -- new 'x', with value 10
       print(x)            --> 10
       x = x+1
       do                  -- another block
         local x = x+1     -- another 'x'
         print(x)          --> 12
       end
       print(x)            --> 11
     end
     print(x)              --> 10  (the global one)
    \end{lstlisting}
  \end{block}
\end{frame}

\frame{%
  \frametitle{Definições e Declarações}
  \begin{block}{}
      \centering
      \begin{itemize}
        \item Lua é dinamicamente tipada.
        \item Não há definições de tipos na linguagem, os valores carregam seu próprio tipo.
      \end{itemize}
  \end{block}
}

\section{Valores e Tipos de Dados}
\frame{%
  \frametitle{Definições e Declarações}
  \begin{block}{}
      \centering
      \begin{itemize}
        \item Lua é dinamicamente tipada.
        \item Não há definições de tipos na linguagem, os valores carregam seu próprio tipo.
      \end{itemize}
  \end{block}
}

\section{Variáveis e Constantes}

\section{Expressões e Comandos}

\end{document}
% vim: tw=80 et ts=2 sw=2 sts=2
