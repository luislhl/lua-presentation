\documentclass{ime-beamer}
\usepackage[portuges]{babel}
\usepackage[utf8]{inputenc}
\usepackage{graphicx}
\usepackage{listings}			% Para usar \lstinputlisting e incluir código
\usepackage{color}
\usepackage{multicol}

\definecolor{dkgreen}{rgb}{0,0.6,0}
\definecolor{gray}{rgb}{0.5,0.5,0.5}
\definecolor{mauve}{rgb}{0.58,0,0.82}

\lstset{frame=tb,
  language=[5.2]Lua,
  showstringspaces=false,
  columns=flexible,
  basicstyle={\tiny\ttfamily},
  numbers=none,
  numberstyle=\tiny\color{gray},
  keywordstyle=\color{blue},
  commentstyle=\color{dkgreen},
  stringstyle=\color{mauve},
  breaklines=true,
  breakatwhitespace=true,
  tabsize=3
}

\title[Computação em nuvem]{%
  Cloud Computiong
}
\author[Luis Helder\and Victor Bramigk]{%
  Luis Helder\\
  Victor Bramigk\\
}

% as imagens ficam nesse diretório
\graphicspath{{img/}}

\begin{document}

\frame{\maketitle}

\frame{%
  \frametitle{Roteiro}
  \tableofcontents
}

\section{Introdução}
\frame{%
  \frametitle{O que é}
  \begin{block}{}
    \centering
    \begin{itemize}
      \item Lua é uma linguagem de extensão projetada para suportar programação
        procedural.
      \item Concebida em 1993 na PUC-Rio, é a única LP desenvolvida num país
        em desenvolvimento a alcançar relevância global.
      \item Como uma linguagem de extensão, Lua necessita de um programa
        anfitrião. Porém, é oferecido um interpretador que funciona
        como programa anfitrião.
      \item Também oferece suporte à programação funcional e à orientada a
        objetos.
      \item É implementada como uma biblioteca escrita em C.
    \end{itemize}
  \end{block}
}

\frame{%
  \frametitle{Método de Implementação}
  \begin{block}{}
      \centering
      \begin{itemize}
        \item Lua é uma linguagem híbrida.
        \item Compilação por padrão é em tempo de execução, mas pode ser feita
          previamente, para aumentar a performance.
        \item VM baseada em registro.
      \end{itemize}
  \end{block}
}

\section{Amarrações}
\begin{frame}[fragile]
  \frametitle{Escopo}
  \begin{block}{}
      \centering
      \begin{itemize}
        \item Toda variável é global, a não ser que seja explicitamente
          declarada como local. 
        \item Utiliza-se 'local' para declaração de variáveis locais.
      \end{itemize}
  \end{block}
  \begin{block}{}
    \begin{lstlisting}
     x = 10                -- global variable
     do                    -- new block
       local x = x         -- new 'x', with value 10
       print(x)            --> 10
       x = x+1
       do                  -- another block
         local x = x+1     -- another 'x'
         print(x)          --> 12
       end
       print(x)            --> 11
     end
     print(x)              --> 10  (the global one)
    \end{lstlisting}
  \end{block}
\end{frame}

\section{Valores e Tipos de Dados}
\frame{%
  \frametitle{Tipos}
  \begin{block}{}
    Lua é dinamicamente tipada. 
    
    Não há definições de tipos na linguagem, os valores carregam seu próprio tipo.
  
    A linguagem define 8 tipos de dados:
    \begin{itemize}
      \item nil
      \item boolean
      \item number (integer/float)
      \item string
      \item function
      \item userdata (bloco de dados do programa em C)
      \item thread
      \item table (único tipo composto)
    \end{itemize}
  \end{block}
}

\frame{%
  \frametitle{Tables}
  \begin{block}{}
    Implementam a ideia de array associativo, ou seja, associam chaves e
        valores. As chaves podem ser qualquer valor menos nil e NaN.

    São o único mecanismo de estrutura de dados em Lua.

    Podem ser usadas para criar:
    \begin{itemize}
      \item Arrays
      \item Records
      \item Sets
      \item Trees
      \item Namespaces
      \item Classes
      \item Etc.
    \end{itemize}
  \end{block}
}

\begin{frame}[fragile]
  \frametitle{Tables - exemplos}
  \begin{block}{}
    \begin{itemize}
      \item Table usado como um Record: 
        \begin{lstlisting}
     point = { x = 10, y = 20 }   -- Create new table
     print(point["x"])            -- Prints 10
     print(point.x)               -- Has exactly the same meaning as line above. The easier-to-read dot notation is just syntactic sugar.
        \end{lstlisting}
      \item Table usado como namespace:
        \begin{lstlisting}
        Point = {}
         
        Point.new = function(x, y)
          return {x = x, y = y}  --  return {["x"] = x, ["y"] = y}
          end
           
          Point.set_x = function(point, x)
            point.x = x  --  point["x"] = x;
            end
        \end{lstlisting}
    \end{itemize}
  \end{block}
\end{frame}

\frame{%
  \frametitle{Metatables}
  \begin{block}{}
    \begin{itemize}
      \item Controlam como objetos se comportam ao serem submetidos a determinados
    eventos, como operações aritméticas, comparações e indexações, ou ao ser
    \emph{garbage collected}.
      \item Uma metatable é uma table associada ao objeto, onde as chaves são
        o nome do evento, e os valores são funções, chamadas de
        \emph{metamethods}, a
        serem chamadas na ocorrência do evento.
      \item No caso de operações binárias, Lua procura no primeiro operando um \emph{metamethod}
        definido para o evento correspondente. Caso não encontre, procura no
        segundo operando.
    \end{itemize}
  \end{block}
}

\frame{%
  \frametitle{Metatables - exemplos de eventos}
  \begin{block}{}
    \begin{itemize}
      \item add - Ocorre ao aplicar-se o operador + ao objeto.
      \item eq - Ocorre ao aplicar-se o perador == ao objeto. 
      \item index - Se o objeto for uma \emph{Table}, ocorre ao tentar-se
        acessar um índice inexistente. Caso contrário, ao tentar-se acessar
        acessar qualquer índice. O \emph{metamethod} pode ser tanto uma
        função quanto uma Table.
      \item call - Ocorre caso o objeto não seja uma \emph{function} e
        tenta-se aplicar o operador de chamada () nele.
    \end{itemize}
  \end{block}
}

\begin{frame}[fragile]
  \frametitle{Metatables - exemplo de uso}
  \begin{block}{}
    Fibonacci em Lua usando-se Tables e Metatables
    \begin{lstlisting}
    fibs = { 1, 1 }                                -- Initial values for fibs[1] and fibs[2].
    setmetatable(fibs, {
      __index = function(values, n)                --[[ __index is a function predefined by Lua, 
                                                        it is called if key "n" does not exist. ]]
        values[n] = values[n - 1] + values[n - 2]  -- Calculate and memorize fibs[n].
        return values[n]
      end
    })
    \end{lstlisting}
  \end{block}
\end{frame}

\section{Variáveis e Constantes}

\section{Expressões e Comandos}
\begin{frame}[fragile]
  \frametitle{Comandos - Atribuição}
  \begin{block}{}
    \begin{itemize}
      \item Lua suporta atribuição múltipla:
        \begin{lstlisting}
          x, y = 2, 3
        \end{lstlisting}
      \item Todas as expressões no comando são avaliadas antes das atribuições serem
        realizadas. O código a seguir altera a[3], não a[4]:
        \begin{lstlisting}
          i = 3
          i, a[i] = i+1, 20
        \end{lstlisting}
      \item Funções podem ser chamadas na lista de valores à direita:
        \begin{lstlisting}
        function f()
          return 2,3
        a,b,c = 1, f()
        \end{lstlisting}
    \end{itemize}
  \end{block}{}
\end{frame}

\begin{frame}[fragile]
  \frametitle{Comandos - Estruturas de controle}
  \begin{block}
    %TODO
  \end{block}
\end{frame}

\end{document}
% vim: tw=80 et ts=2 sw=2 sts=2
