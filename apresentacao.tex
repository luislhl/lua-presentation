\documentclass{ime-beamer}
\usepackage[portuges]{babel}
\usepackage[utf8]{inputenc}
\usepackage{graphicx}
\usepackage{listings}			% Para usar \lstinputlisting e incluir código
\usepackage{multicol}

\title[Computação em nuvem]{%
  Cloud Computiong
}
\author[Luis Helder\and Victor Bramigk]{%
  Luis Helder\\
  Victor Bramigk\\
}

% as imagens ficam nesse diretório
\graphicspath{{img/}}

\begin{document}
\frame{\maketitle}

\frame{%
  \frametitle{Roteiro}
  \tableofcontents
}

\section{Introdução}
\frame{%
  \frametitle{O que é}
  \begin{block}{}
    \centering
    \begin{itemize}
      \item Lua é uma linguagem de extensão projetada para suportar programação
        procedural.
      \item Concebida em 1993 na PUC-Rio, é a única LP desenvolvida num país
        em desenvolvimento a alcançar relevância global.
      \item Como uma linguagem de extensão, Lua necessita de um programa
        anfitrião. Porém, é oferecido um interpretador que funciona
        como programa anfitrião.
      \item Também oferece suporte à programação funcional e à orientada a
        objetos.
      \item É implementada como uma biblioteca escrita em C.
    \end{itemize}
  \end{block}
}

\frame{%
  \frametitle{Método de Implementação}
  \begin{block}{}
      \centering
      \begin{itemize}
        \item Lua é uma linguagem híbrida.
        \item Compilação por padrão é em tempo de execução, mas pode ser feita
          previamente, para aumentar a performance.
        \item VM baseada em registro.
      \end{itemize}
  \end{block}
}

\section{Valores e Tipos de Dados}

\section{Expressões e Comandos}

\end{document}
% vim: tw=80 et ts=2 sw=2 sts=2
