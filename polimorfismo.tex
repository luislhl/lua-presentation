\begin{frame}[fragile]
  \frametitle{Polimorfismo}
  \begin{block}{}
    \centering
    Formas: coerção, sobrecarga e paramétrico.\\

    A coerção é feita quando se opera um operando de um tipo e um valor
    de outro tipo é fornecido, desde que esse tipo possa ser convertido.
    Exemplo:
    
  \begin{block}{}
    \begin{lstlisting}
        print (2 ^ “3”) –> 6
    \end{lstlisting}
  \end{block}
    
    A string “3” é convertida para o inteiro 3,
    tornando a operação possível.
  \end{block}
\end{frame}

\begin{frame}[fragile]
  \frametitle{Polimorfismo}
  \begin{block}{}
    \centering
Não é necessário declarar o tipo dos parâmetros formais. 
Exemplo, a função table.insert adiciona um valor de qualquer
tipo a uma tabela (exceto nil).
  \end{block}
\end{frame}

\begin{frame}[fragile]
  \frametitle{Polimorfismo}
  \begin{block}{}
    \centering
    \begin{lstlisting}
    t = {}
    table.insert( t, “10” )
    table.insert( t, 10 )
    table.insert( t, function() return 10 end )
    for k, v in pairs( t ) do
       print( type(v), v )
    end
    Imprimirá:
    string 10
    number 10
    function     function: 0xa037158
    \end{lstlisting}
  \end{block}
\end{frame}
